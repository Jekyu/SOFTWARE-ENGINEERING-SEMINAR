\section{Introduction}
%1 page. Context, challenges, prior work.
%An example of citation is: Information theory was first formalized by Shannon in 1948

Parking management is a significant challenge in urban and corporate spaces. Traditional methods, which often rely on manual records, can lead to inefficiencies, errors, and delays. These issues affect the accessibility and availability of parking spaces, leading to poor operational management. Furthermore, manually recording vehicle entries and exits makes it difficult to trace vehicles accurately, impacting security and billing.\cite{chien2020lowcost}.


In response to this problem, we propose the development of a low-cost digital application designed to optimize the vehicle entry and exit registration process. The proposed solution is designed to be cost-effective, allowing smaller parking lots to benefit from technology without a large financial investment. By streamlining the process of vehicle registration and receipt generation, our system can improve operational efficiency, reduce human error, and provide real-time tracking of vehicles within the parking lot.

This paper describes the design and implementation of the proposed parking management system. We outline its architecture, the technology stack used, and the testing methods employed to evaluate the system’s effectiveness. The results show that the system successfully automates parking management and reduces errors compared to traditional manual methods.